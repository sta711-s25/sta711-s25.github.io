\documentclass[11pt]{article}
\usepackage{url}
\usepackage{alltt}
\usepackage{bm}
\usepackage{bbm}
\linespread{1}
\textwidth 6.5in
\oddsidemargin 0.in
\addtolength{\topmargin}{-1in}
\addtolength{\textheight}{2in}

\usepackage{amsmath}
\usepackage{amssymb}
\usepackage{bbm}

\begin{document}


\begin{center}
\Large
STA 711 Exam 1\\
\normalsize
\vspace{5mm}
\end{center}

\noindent \textbf{Due:} Monday, March 3, 10:00pm on Canvas.\\ 

\noindent \textbf{Instructions:} Submit your work as a single PDF. For this assignment, you may include written work by scanning it and incorporating it into the PDF. Include all R code needed to reproduce your results in your submission.\\

\noindent \textbf{Mastery:} To master this exam, you will need to master at least 5 of the 7 questions. (You may, of course, attempt all 7).\\

\noindent \textbf{Rules:} This is an open-book, open-notes exam. You may:
\begin{itemize}
\item Use any resources from the course (the textbook, the course website, class notes, previous assignments, etc.)
\item Email me, or come to office hours, with specific questions (I may be somewhat less helpful than for regular assignments)
\item Use one or two days from your bank of extension days, if you need more time on the exam
\end{itemize}
You may \textit{not}:
\begin{itemize}
\item Use the internet to look up any questions on the exam
\item Use any resources outside of the course (other textbooks, any textbook solution manuals, notes from other courses or universities, etc.)
\item Use WolframAlpha or any generative AI to help solve the problems
\item Discuss the exam with anyone else
\end{itemize}

\newpage

\section*{Maximum likelihood questions}

\begin{enumerate}
\item Let $Y_1,...,Y_n \overset{iid}{\sim} Uniform(a, b)$, where $a$ and $b$ are unknown and $a < b$. Recall that a uniform distribution has pdf
\begin{align*}
f(y|a,b) = \begin{cases}
\frac{1}{b - a} & a \leq y \leq b \\
0 & \text{else}
\end{cases}
\end{align*}
\begin{enumerate}
\item Find the maximum likelihood estimators $\widehat{a}$ and $\widehat{b}$.
\item Let $\tau = \mathbb{E}[Y_1]$. Find the MLE $\widehat{\tau}$.
\end{enumerate}

\bigskip

\item Let $Y_1,...,Y_n$ be iid from a distribution with pdf
$$f(y|\lambda) = \frac{2}{\lambda \sqrt{2 \pi}} e^y \exp \left\lbrace \frac{-(e^y - 1)^2}{2\lambda^2} \right\rbrace,$$
where $y > 0$ and $\lambda > 0$. Find the MLE of $\lambda$.

\bigskip

\item Let $Y_1,...,Y_n$ be an iid sample from a continuous distribution with pdf 
$$f(y|\theta) = \frac{1}{2} \exp\{-|y - \theta|\},$$
where $-\infty < y < \infty$ and $-\infty < \theta < \infty$. Find the maximum likelihood estimator of $\theta$. You are not required to check a second derivative for this problem. 

\bigskip

\item Let $Y_1,...,Y_n$ be iid from a distribution with pdf
$$f(y | \theta) = a^{\theta} \theta y^{-\theta - 1}$$
where $\theta > 0$, $y \geq a$, and $a$ is a known constant. Find the MLE of $\theta$.

\bigskip

\item Let $Y_1,...,Y_n$ be iid from a distribution with pdf
$$f(y | \theta_1, \theta_2) = \begin{cases} \dfrac{1}{\theta_1 + \theta_2} \exp\left\lbrace \dfrac{-y}{\theta_1}\right\rbrace & y > 0 \\ \dfrac{1}{\theta_1 + \theta_2} \exp\left\lbrace \dfrac{y}{\theta_2}\right\rbrace & y \leq 0 \end{cases}$$
with $\theta_1, \theta_2 > 0$. Show that the maximum likelihood estimators of $\widehat{\theta}_1$ and $\widehat{\theta}_2$ are given by
$$\widehat{\theta}_1= T_1 + \sqrt{T_1 T_2} \hspace{1cm} \widehat{\theta}_2 = T_2 + \sqrt{T_1 T_2}$$
where
$$T_1 = \frac{1}{n} \sum \limits_{i=1}^n Y_i \mathbbm{1}\{Y_i > 0\} \hspace{1cm} T_2 = -\frac{1}{n} \sum \limits_{i=1}^n Y_i \mathbbm{1}\{Y_i \leq 0\}.$$
You are not required to check a second derivative for this problem.

\end{enumerate}

\newpage

\section*{Exponential regression}

The exponential distribution with parameter $\lambda$ has pdf

$$f(y | \lambda) = \frac{1}{\lambda} \exp\left\lbrace - \frac{y}{\lambda} \right\rbrace, \hspace{1cm} y > 0.$$

\begin{enumerate}
\item[6.] 
\begin{enumerate}
\item Show that this exponential distribution is an example of an exponential dispersion model (EDM), by finding the canonical parameter $\theta$, the dispersion parameter $\phi$, and the cumulant function $\kappa(\theta)$.

\item Suppose $Y \sim Exponential(\lambda)$ with the pdf above. Using properties of the cumulant function $\kappa$, calculate $\mathbb{E}[Y]$ and $Var(Y)$. (Your answer should use $\kappa$ to calculate these values; do not integrate the pdf or derive the mgf).

\item Let $Y_i > 0$ be a continuous, positive response variable of interest, and let $X_i = (1, X_{i,1},...,X_{i,p})^T \in \mathbb{R}^{p+1}$ be a vector of covariates. Suppose we observe independent samples $(X_1, Y_1),...,(X_n, Y_n)$ from the following model:

\begin{align*}
Y_i &\sim Exponential(\lambda_i) \\
-\frac{1}{\lambda_i} &= \beta^T X_i
\end{align*}

Write down the score function $U(\beta | X, Y)$ and the Hessian ${\bf H}(\beta | X, Y)$ in matrix form. Your answer should involve the $\lambda_i$ (i.e. do not leave the answer in terms of $\mu_i$, $\theta_i$, $Var(Y_i)$, etc.)
\end{enumerate}

\item[7.] Now we apply the model from Question 6 to real data. A factory is interested in the relationship between the amount of stress applied to a piece of steel, and the time it takes until that steel breaks. We use the following model:
\begin{align*}
time_i &\sim Exponential(\lambda_i) \\
-\dfrac{1}{\lambda_i} &= \beta_0 + \beta_1 stress_i
\end{align*}
The raw data contains $n = 40$ observations $(stress_1, \ time_1),...,(stress_{40}, \ time_{40})$.\\

\noindent You can load the data into R by
\begin{verbatim}
steel <- read.csv("https://sta711-s25.github.io/exams/steel.csv")
\end{verbatim}

Use Newton's method to calculate $\widehat{\beta} = (\widehat{\beta}_0, \widehat{\beta}_1)^T$. Begin at 

$$\beta^{(0)} = \begin{pmatrix}
- \dfrac{1}{\overline{Y}} \\ 0
\end{pmatrix} = \begin{pmatrix}
-15.18397 \\ 0
\end{pmatrix}$$

For the purpose of this question, stop when 
$$\max \{ |\beta_0^{(r+1)} - \beta_0^{(r)}|, \ |\beta_1^{(r+1)} - \beta_1^{(r)}| \} < 0.0001$$

\end{enumerate}

\end{document}
