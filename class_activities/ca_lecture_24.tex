\documentclass[11pt]{article}
\usepackage{url}
\usepackage{alltt}
\usepackage{bm}
\usepackage{bbm}
\linespread{1}
\textwidth 6.5in
\oddsidemargin 0.in
\addtolength{\topmargin}{-1in}
\addtolength{\textheight}{2in}

\usepackage{amsmath}
\usepackage{amssymb}
\usepackage{hyperref}
\usepackage{bbm}

\newcommand{\indep}{\perp \!\!\! \perp}

\begin{document}


\begin{center}
\Large
Asymptotic distribution of the LRT \\
\normalsize
\vspace{5mm}
\end{center}

\noindent Suppose we observe iid data $X_1,...,X_n$ from a distribution with parameter $\theta \in \mathbb{R}$, and we wish to test $H_0: \theta = \theta_0$ vs. $H_A: \theta \neq \theta_0$.\\

\noindent \textbf{Theorem:} Under $H_0$,

$$2 \log \left( \frac{L(\widehat{\theta}_{MLE} | {\bf X})}{L(\theta_0 | {\bf X})} \right) \overset{d}{\to} \chi^2_1$$

\subsection*{Key proof pieces}

\begin{enumerate}
\item Let $\ell(\theta) = \log L(\theta | {\bf X})$ denote the log-likelihood. Using a second-order Taylor expansion of $\ell(\theta_0)$ around $\widehat{\theta}$, argue that if $\widehat{\theta}_{MLE}$ is close to $\theta_0$ then
$$2 \ell(\widehat{\theta}) - 2 \ell(\theta_0) \approx -\ell''(\widehat{\theta})(\widehat{\theta} - \theta_0)^2 = -\frac{1}{n} \ell''(\widehat{\theta}) (\sqrt{n}(\widehat{\theta} - \theta_0) )^2$$

\vspace{7cm}

\item Using results previously derived (when proving the asymptotic normality of the MLE), find the limits for the following two quantities when $H_0$ is true:

\begin{itemize}
\item $-\frac{1}{n} \ell''(\widehat{\theta}) \overset{p}{\to}$
\bigskip
\item $\sqrt{n}(\widehat{\theta} - \theta_0) \overset{d}{\to}$
\end{itemize}

\newpage


\item Apply Slutsky's theorem and the continuous mapping theorem to argue that

$$2 \ell(\widehat{\theta}) - 2 \ell(\theta_0) \overset{d}{\to} \chi^2_1$$

\end{enumerate}

\end{document}
