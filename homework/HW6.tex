\documentclass[11pt]{article}
\usepackage{url}
\usepackage{alltt}
\usepackage{bm}
\usepackage{bbm}
\linespread{1}
\textwidth 6.5in
\oddsidemargin 0.in
\addtolength{\topmargin}{-1in}
\addtolength{\textheight}{2in}

\usepackage{amsmath}
\usepackage{amssymb}
\usepackage{hyperref}
\usepackage{bbm}

\begin{document}


\begin{center}
\Large
STA 711 Homework 6\\
\normalsize
\vspace{5mm}
\end{center}

\noindent \textbf{Due:} Friday, March 21, 10:00pm on Canvas.\\ 

\noindent \textbf{Instructions:} Submit your work as a single PDF. You may choose to either hand-write your work and submit a PDF scan, or type your work using LaTeX and submit the resulting PDF. See the course website for a \href{https://sta711-s25.github.io/homework/hw_template.tex}{homework template file} and \href{https://sta711-s25.github.io/homework/latex_instructions/}{instructions} on getting started with LaTeX and Overleaf.

\section*{Multivariate normal distributions}

\noindent The multivariate normal distribution will appear frequently in 711, for example as the asymptotic distribution of our coefficient estimates $\widehat{\beta}$. The purpose of this section is to derive a few important properties of this distribution.\\

\noindent To begin, let us formally define the multivariate normal. There are many equivalent definitions (one person's definition is another's ``if and only if'' theorem!). We will define the multivariate normal by first defining a multivariate standard normal, and then defining other multivariate normals as a transformation.\\

\noindent \textbf{Definition (multivariate standard normal):} Let $Z = (Z_1,...,Z_k)^T$ be a $k$-dimensional random vector. We say that $Z \sim N(\bm{0}, {\bf I})$ (the multivariate normal with mean vector $\bm{0} \in \mathbb{R}^k$ and identity covariance matrix ${\bf I} \in \mathbb{R}^{k \times k}$) if
$$Z_1,...,Z_k \overset{iid}{\sim} N(0, 1)$$
That is, the components are iid univariate standard normals.\\

\noindent \textbf{Definition (multivariate normal):} Let $X = (X_1,...,X_k)^T$. We say that $X \sim N(\bm{\mu}, \bm{\Sigma})$ if there exists $Z = (Z_1,...,Z_k)^T$ such that $Z \sim N(\bm{0}, {\bf I})$ and
$$X =\bm{\mu} + {\bf A} Z$$
where $\bm{\Sigma} = {\bf A} {\bf A}^T$.\\


\begin{enumerate}

\item Let $X \in \mathbb{R}^k$ such that $X \sim N(\bm{\mu}, \bm{\Sigma})$, and let ${\bf a} \in \mathbb{R}^m$ and ${\bf B} \in \mathbb{R}^{m \times k}$. Using the definitions above, show that
$${\bf a} + {\bf B} X \sim N({\bf a} + {\bf B} \bm{\mu}, {\bf B} \bm{\Sigma} {\bf B}^T)$$


\item Now let us prove some further properties of multivariate normal distributions:

\begin{enumerate}

\item Let $X \sim N(\bm{\mu}, \bm{\Sigma})$, where $X \in \mathbb{R}^q$. Suppose that for some $1 \leq p < q$, $\bm{\Sigma}$ can be partitioned as
$$\bm{\Sigma} = \begin{pmatrix}
\bm{\Sigma}_{11} & \bm{0}_{p \times (q - p)} \\
\bm{0}_{(q - p) \times p} & \bm{\Sigma}_{22}
\end{pmatrix},$$
where $\bm{\Sigma}_{11}$ is $p \times p$, $\bm{\Sigma}_{22}$ is $(q - p) \times (q - p)$, and $\bm{0}_{m \times n}$ denotes the matrix of zeros of the specified dimensions. Similarly partition 
$$X = \begin{pmatrix}
X_{(1)} \\
X_{(2)}
\end{pmatrix} \hspace{1cm} \bm{\mu} = \begin{pmatrix}
\bm{\mu}_{(1)} \\
\bm{\mu}_{(2)}
\end{pmatrix},$$
into vectors of length $p$ and $q - p$. Prove that 
$$X_{(1)} \sim N(\bm{\mu}_{(1)}, \bm{\Sigma}_{11}), \hspace{0.5cm} X_{(2)} \sim N(\bm{\mu}_{(2)}, \bm{\Sigma}_{22}),$$
and $X_{(1)}$ and $X_{(2)}$ are independent.

\item Using (a), conclude that if $X = (X_1,...,X_q)^T \sim N(\bm{\mu}, \bm{\Sigma})$, then the entries $X_i$ and $X_j$ are independent \textit{if and only} if $\bm{\Sigma}_{ij} = Cov(X_i, X_j) = 0$. (That is, while 0 covariance does not usually imply independence, it \textit{does} for entries of a multivariate normal).

\end{enumerate}


\end{enumerate}


\section*{Logistic regression with earthquake data}

In the second part of this assignment, you will work with a dataset from DrivenData, an online data competition site that hosts competitions aimed at improving education, health, safety, and general well being for individuals around the world.\\

\noindent Our data come from the 2015 Gorkha earthquake in Nepal. After the earthquake, a large scale survey was conducted to determine the amount of damage the earthquake caused for homes, businesses and other structures. This is one of the largest post-disaster surveys in the world, and researchers are interested in which building characteristics are associated with earthquake damage.\\

\noindent You will work with a subset of the earthquake data, consisting of 211774 buildings, containing the following variables:
\begin{itemize}
\item \verb;Damage;: whether the building sustained any damage (1) or not (0)

\item \verb;Age;: the age of the building (in years)

\item \verb;Surface;: a categorical variable recording the surface condition of the land around the building. There are three different levels: \verb;n;, \verb;o;, and \verb;t;. (The researchers who collected the data anonymized the level names to protect inhabitants' privacy).
\end{itemize}

\noindent You can load the data into R by
\begin{verbatim}
earthquake <- read.csv("https://sta711-s25.github.io/homework/earthquake_small.csv")
\end{verbatim}

\noindent You will work with the following logistic regression model (you may assume all assumptions are met; no transformations or diagnostics are needed):
$$Damage_i \sim Bernoulli(p_i)$$
$$\log \left( \dfrac{p_i}{1 - p_i} \right) = \beta_0 + \beta_1 Age_i + \beta_2 SurfaceO_i + \beta_3 SurfaceT_i + \beta_4 Age_i \cdot SurfaceO_i + \beta_5 Age_i \cdot SurfaceT_i$$

where $SurfaceO$ and $SurfaceT$ are indicator variables for whether surface is o or t, respectively.\\

\begin{enumerate}
\item[3.] 
\begin{enumerate}
\item Fit the logistic regression model in R, and interpret the estimated slope $\widehat{\beta}_1$ in terms of the \textit{odds} of damage.

\item Calculate the estimated probability of damage for a 50 year old building with surface condition = t.
\end{enumerate}

\item[4.] The researchers want to know whether the relationship between Age and the probability of damage is the same for buildings in all three surface conditions. Use a Wald test to address the researchers' question; you should state the hypotheses in terms of one or more model parameters, calculate a test statistic and p-value, and make a conclusion.

\end{enumerate}

\section*{Global F-test}

Suppose that $V_1 \sim \chi^2_{d_1}$ and $V_2 \sim \chi^2_{d_2}$ are independent $\chi^2$ random variables. Then 
$$F = \dfrac{V_1/d_1}{V_2/d_2} \sim F_{d_1, d_2}$$
where $F_{d_1, d_2}$ denotes the $F$-distribution with numerator degrees of freedom $d_1$ and denominator degrees of freedom $d_2$.\\

\noindent The $F$-distribution is important for hypothesis testing in linear regression models. Suppose we observe independent data $(X_1, Y_1),...,(X_n, Y_n)$, where $Y_i = \beta^T X_i + \varepsilon_i$, with $\beta = (\beta_0, ..., \beta_k)^T$ and $\varepsilon_i \overset{iid}{\sim} N(0, \sigma^2)$. We wish to test the hypotheses
$$H_0: \beta_1 = \cdots = \beta_k = 0 \hspace{1cm} H_A: \text{at least one of } \beta_1,...,\beta_k \neq 0.$$
The $F$-test for these hypotheses is based on the $F$-statistic
$$F = \dfrac{(SSTO - SSE)/k}{SSE/(n - k - 1)},$$
where $F \sim F_{k, n-k-1}$ under $H_0$, and
\begin{align*}
SSTO = \sum \limits_{i=1}^n (Y_i - \overline{Y})^2 \hspace{1cm} SSE = \sum \limits_{i=1}^n (Y_i - \widehat{Y}_i)^2
\end{align*}
where 
$$\widehat{Y} = \begin{pmatrix}
\widehat{Y}_1 \\ \vdots \\ \widehat{Y}_n
\end{pmatrix} = {\bf X}({\bf X}^T{\bf X})^{-1} {\bf X}^T Y$$

\noindent The goal of this part of the assignment is to demonstrate that, indeed, $F \sim F_{k, n-k-1}$ under $H_0$.

\begin{enumerate}

\item[5.] The matrix ${\bf H} = {\bf X}({\bf X}^T{\bf X})^{-1} {\bf X}^T$ is often called the \textit{hat} matrix (because it puts a ``hat'' on $Y$). Show that ${\bf H}$ is idempotent.

\item[6.] The rank of an idempotent matrix is equal to its trace. Using this fact, show that 
$$\text{rank}({\bf H}) = k+1$$

\item[7.] Let ${\bf J}_n$ be the $n \times n$ matrix of all 1s. Using the fact that ${\bf H} {\bf X} = {\bf X}$, argue that ${\bf H} {\bf J}_n = {\bf J}_n$.

\item[8.] Using the previous question, show that ${\bf H} - \frac{1}{n} {\bf J}_n$ is idempotent.

\item[9.] Show that

$$SSTO = SSE + \sum \limits_{i=1}^n (\widehat{Y}_i - \overline{Y})^2$$

\item[10.] Show that

$$\sum \limits_{i=1}^n (Y_i - \beta_0)^2 = SSE + \sum \limits_{i=1}^n (\widehat{Y}_i - \overline{Y})^2 + n (\overline{Y} - \beta_0)^2$$


\item[11.] Show that under $H_0$, $\frac{1}{\sigma^2}\sum \limits_{i=1}^n (Y_i - \beta_0)^2 \sim \chi^2_n$.

\item[12.] Using the above questions, find symmetric matrices $A_1, A_2, A_3$ such that under $H_0$, 
$$\frac{1}{\sigma^2}\sum \limits_{i=1}^n (Y_i - \beta_0)^2 = Z^T A_1 Z + Z^T A_2 Z + Z^T A_3 Z$$
where $Z \sim N(\bm{0}, {\bf I})$, $\frac{1}{\sigma^2} SSE = Z^T A_1 Z$, and $\frac{1}{\sigma^2} (SSTO - SSE) = Z^T A_2 Z$.

\item[13.] Using the matrices $A_1, A_2, A_3$ from the previous question, show that $rank(A_1) = n-k-1$, $rank(A_2) = k$, and $rank(A_3) = 1$.

\item[14.] By applying Cochran's theorem, show that $F = \dfrac{(SSTO - SSE)/k}{SSE/(n - k - 1)} \sim F_{k, n-k-1}$ under $H_0$.
\end{enumerate}

\end{document}
